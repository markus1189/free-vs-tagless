 \documentclass[aspectratio=169]{beamer}

% Must be loaded first
\usepackage{tikz}
\usetikzlibrary{shapes,arrows,positioning,fit,calc}

\usepackage[utf8]{inputenc}
\usepackage{textpos}

% Font configuration
\usepackage{fontspec}

\usepackage[listings, minted]{tcolorbox}
\usepackage{smartdiagram}
\usepackage[export]{adjustbox}

\input{font.tex}

% Tikz for beautiful drawings
\usetikzlibrary{mindmap,backgrounds}
\usetikzlibrary{arrows.meta,arrows}
\usetikzlibrary{shapes.geometric}

% Minted configuration for source code highlighting
\usepackage{minted}
\setminted{highlightcolor=orange!50, linenos}
\setminted{style=lovelace}

\usepackage[listings, minted]{tcolorbox}
\tcbset{left=6mm}

% Use the include theme
\usetheme{codecentric}

% Metadata
\title{Free vs Final Tagless}
\author{Markus Hauck (@markus1189)}

% The presentation content
\begin{document}

\begin{frame}[noframenumbering,plain]
  \titlepage{}
\end{frame}

\section{Introduction}\label{sec:introduction}

\begin{frame}
  \begin{center}
    {\Huge Composition\\}
    (how to do more with less)
  \end{center}
\end{frame}

\begin{frame}[fragile]
  \frametitle{Content}
  \begin{center}
    \resizebox{!}{0.8\textheight}{
      \smartdiagram[bubble diagram]{
        Content,
        2 Monoids,
        1 Introduction,
        5 Conclusion,
        4 Streaming,
        3 Applicatives
      }
    }
  \end{center}
\end{frame}

\begin{frame}
  \frametitle{The Toy Language}
  \begin{itemize}
  \item toy language with operations
  \item integer constants
  \item integer addition
  \item string constants
  \item string concatenation
  \end{itemize}
\end{frame}

\section{Free}\label{sec:free}

\begin{frame}
  \begin{itemize}
  \item also: tagged initial encoding
  \item ``tagged'' because sum types are represented at runtime as
    pairs (tag, payload), the tag is used in ``eval'' to do pattern
    matching
  \item use data structure to capture the program
  \item traverse this structure using interpreter
  \end{itemize}
\end{frame}

\begin{frame}
  \frametitle{Tags}
  \begin{itemize}
  \item untyped representation that allows invalid constants
  \item GADT makes wrong expression not representable
  \item tagged and tagless initial encoding possible
  \end{itemize}
\end{frame}

\begin{frame}
  \frametitle{Free \textemdash{} Pros and Cons}
  \begin{itemize}
  \item program reified as a data structure
  \item allows inspection and transformation
  \item can be serialized
  \end{itemize}

  \section{Finally Tagless}\label{sec:finally-tagless}

\begin{frame}
  \begin{itemize}
  \item use overloaded functions (all the way)
  \item use denotation in semantic algebra and not AST
  \end{itemize}
\end{frame}

\begin{frame}
  \frametitle{Expression Problem}
  \begin{itemize}
  \item extremely nice property of finally tagless
  \item extensibility in two dimensions
  \item neat way to solve expression problem!
  \end{itemize}
\end{frame}

\end{frame}

\section{Optimization And Inspection}\label{sec:optimization-and-inspection}

\begin{frame}
  \frametitle{Optimization}
  \begin{itemize}
  \item time to talk about program optimization and transformation
  \item start with Free, then finally tagless
  \item myth: not possible in finally tagless, which is wrong
  \item instantiate interpreter that keeps track of information that is passed down
  \end{itemize}
\end{frame}

\begin{frame}
  \frametitle{Analyzing Programs}
  \begin{itemize}
  \item in both approaches: level of introspection depends
  \item if you use Monad, very limited
  \item the more you represent, the more power
  \end{itemize}
\end{frame}

\section{Free vs Finally Tagless}\label{sec:free-vs-tagless}

\begin{frame}
  \frametitle{Free vs Finally Tagless}
  \begin{itemize}
  \item seen both approaches
  \item but important question: when to use which
  \item spoiler: it depends
  \end{itemize}
\end{frame}

\begin{frame}
  \frametitle{When To Use Free}
  \begin{itemize}
  \item doing a lot of transformation/optimization
  \item ability to serialize/deserialize program
  \item doing partial evaluation
  \end{itemize}
\end{frame}

\begin{frame}
  \frametitle{When To Use Finally Tagless}
  \begin{itemize}
  \item programs are inspected max once
  \item executed a lot more than inspected
  \item no need to keep around the data structure at all
  \end{itemize}
\end{frame}

\begin{frame}
  \frametitle{When To Use Both}
  \begin{itemize}
  \item saw: reason for either one
  \item can we have our cake and eat it, too?
  \end{itemize}
\end{frame}

\begin{frame}
  \frametitle{Converting from Free to Finally Tagless}
\end{frame}

\begin{frame}
  \frametitle{Converting from Finally Tagless to Free}
\end{frame}

\section{Conclusion}\label{sec:conclusion}

\begin{frame}
  \frametitle{Conclusion}
  \begin{itemize}
  \item flexible and \texttt{composable} way to cacluate metrics over text
  \item using \texttt{Monoid} and \texttt{Applicative}
  \item works with iteration and streaming
  \item Principle Of Least Power: sometimes Monads are overrated
  \end{itemize}
\end{frame}

\begin{frame}
  \frametitle{The End}
  \begin{itemize}
  \item \hyperref[sec:introduction]{Introduction}
  \item \hyperref[sec:monoids]{Monoids}
  \item \hyperref[sec:applicatives]{Applicatives}
  \item \hyperref[sec:streaming]{Streaming}
  \item \hyperref[sec:conclusion]{Conclusion}
  \end{itemize}
  \vfill
  \begin{center}
    \url{https://github.com/markus1189/flatmap-beautiful-composition}
  \end{center}
\end{frame}

\appendix{}

\end{document}
